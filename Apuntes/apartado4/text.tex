\section{Apartado 4 - CSS para contenido}

\subsection{Colores}

Los colores en CSS son manejados en hexadecimal y en rgb, la nomenclatura para el uso de
colores en "hexadecimal" esta compuesta por 6 digitos 00 00 00 y estos empiezan del 0 al f,
es decir 16 valores, del 0 al 9 y de la "a" a la "f".
Para agregar color al contenido de un body, se realiza como en el codigo Listing(\ref{css:colores-hexadecimal})
\begin{lstlisting}[caption={Colores en css con hexadecimal},label={css:colores-hexadecimal},language=css]
    body{
    color:#212121;
    }
\end{lstlisting}
Los colores alfa o canal alfa, es la opacidad de un pixel en una imagen, es decir actua
como una máscara de transparencía. Para realizar esto en css, se agregan 2 valores al final,
para obtener un resultado total transparente el valor es 00, un grado de opacidad seria 99 y para que quede 
 totalmente cubierto se agrega ff, asi como se observa en el codigo Listing(\ref{css:colores-hexadecimal-alfa})
 \begin{lstlisting}[caption={Colores en css con hexadecimal},label={css:colores-hexadecimal-alfa},language=css]
    
    body{
        color:#21212100; /* transparente "00"*/
        }

    body{
    color:#21212199; /* grado de opacidad "99"*/
    }
    
    body{
    color:#212121ff; /* cubierto total "ff"*/
    }
\end{lstlisting}
Para hacer uso de rgb, se utiliza la funcion rgb() la cual lleva tres parametros, red, green, blue,
estos estan definidos del 0 al 255, aplicando la funcion con los parametros se veria algo así
\textbf{rgb(0,0,0)}, es decir el valor 255 en el primer parametro equivale a rojo,
el segundo parametro con valor 255 equivale a verde y el ultimo parametro con el valor
255 equivale a azul. Un ejemplo del uso de esta funcion se aprecia en el codigo Listing(\ref{css:colores-rgb})
\begin{lstlisting}[caption={Colores en css con hexadecimal},label={css:colores-rgb},language=css]
    body{
    color:rgb(0,255,255);
    }
\end{lstlisting}
y para agregar el canal alfa con hexadecimal, se utiliza
lla funcion rgba, de la misma manera se agrega un parametro,
en este caso solo es 1 y puede ser de 0 a 255, es decir
de 0 es totalmente transparente y 255 totalmente solido. 
Un ejemplo deluso de esta funcion se aprecia en el codigo Listing(\ref{css:colores-rgba})

\begin{lstlisting}[caption={Colores en css con hexadecimal},label={css:colores-rgba},language=css]
    body{
    color:rgba(0,0,255,2);
    }
\end{lstlisting}

Por norma los vinculos o los anclas <a>, no cambian al heredar el color, en este
caso body, es decir no se utiliza el \textbf{\textit{"selector de grupo"}}, esto debido
a que si se desean realizar cambios a body afectara en otras caracteristicas a la ancla.
Separamos el estilo como se aprecia en el codigo Listing(\ref{css:separar-body-ancla})

\begin{lstlisting}[caption={Colores en css con hexadecimal},label={css:colores-rgba},language=css]
    body{
        color:#212121ff;
    }
    a{
        color:#212121ff;
    }
\end{lstlisting}


\subsection{Valores y unidades}


Existen 2 tipos de valores relativos y absolutos.

El valor de un pixel, es la unidad minima de tu pantalla, y este es parte de los valores absolutos.
\newline
¿Cuantos pixeles tiene tu monitor? fullhd 1920 x 1080, hd 1024x720,
regularmente una laptop tiene 1440 x 720 o 1730 ancho x 768 alto.

Es importante conocer el tamaño de los equipos para saber que tipos de valores se utilizaran.


\textbf{ unidades de medida: }

\begin{itemize}
    \item em
    \item ex
    \item ch
    \item rem
    \item lh
    \item vw
    \item vh
    \item vmin
    \item vw
    \item vmax
\end{itemize}

\textbf{ unidades de medida: }

\begin{itemize}
    \item cm
    \item mm
    \item Q
    \item in
    \item pc
    \item pt
    \item px
\end{itemize}
