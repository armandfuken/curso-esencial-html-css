\section{Apartado 4 - CSS para contenido}

\subsection{Colores}

Los colores canal alpha se agregan 2 ultimos valores para darle elvalor de transparencía de  00 a ff
\begin{lstlisting}[caption={Colores en css con hexadecimal},label={css: colores-hexadecimal},language=css]
    color:#212121ff;
\end{lstlisting}

\begin{lstlisting}[caption={Colores en css con hexadecimal},label={css: colores-rgb},language=css]
    color: rgb(0, 255, 255);
\end{lstlisting}


canal alpha se usa rgba y en el ultimo parametro se agrega 
el valor de transparencía de 0 a 255
\begin{lstlisting}[caption={Colores en css con hexadecimal},label={css: colores-rgba},language=css]
    color: rgba(0, 0, 255, 2)
\end{lstlisting}
s

Los a <a>(ancla) no cambian al heredar el color de body 
no se utiliza el selector de grupo, pero no porque afectaria 
al ancla en futuros cambios, por lo que se separan
